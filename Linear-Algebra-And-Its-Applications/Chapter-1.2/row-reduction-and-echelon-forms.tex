\documentclass{beamer}
\usepackage{mathtools}
\usepackage{amsmath}

\title{Math Notes: Linear Algebra and its Applications}
\subtitle[short subtitle]{Row Reduction and Echelon Forms}
\author{Demitri Swan}
\beamertemplatenavigationsymbolsempty
\date{}

\begin{document}

\begin{frame}
  \titlepage % Generate the title slide
\end{frame}

\begin{frame}{Row Reduction Algorithm: Introduction}
\begin{definition}
A \textbf{row reduction algorithm} is a procedure that transforms any matrix into a state
that can be easily analyzed. This may include:
\begin{itemize}
\item Determining if there is a non-null solution set
\item Determining if there are zero, one, or infinite solutions
\end{itemize}
\end{definition}
\end{frame}

\begin{frame}{Non-Zero Row}
\begin{definition}
A \textbf{non-zero row} is a row that has at least one element whose value is not zero. \\
\begin{block}{Propositional Function}
\begin{itemize}
\item Given a m x n matrix A where m is the row size and n is the column size
\item Given some identifying row index i
\end{itemize}
\begin{equation}
NZR(i, j) = \exists j \in \{ j | 0 <= j < n \} \left( A\left[i, j\right] \neq 0 \right)
\end{equation}
\end{block}
\end{definition}
\end{frame}

\begin{frame}{Non-Zero Column}
\begin{definition}
A \textbf{non-zero column} is a column that has at least one element whose value is not zero. \\
\begin{block}{Propositional Function}
\begin{itemize}
\item Given a m x n matrix A where m is the row size and n is the column size
\item Given some identifying column index j
\end{itemize}
\begin{equation}
NZC(i, j) = \exists i \in \{ i | 0 <= i < m \} \left( A\left[i, j\right] \neq 0 \right)
\end{equation}
\end{block}
\end{definition}
\end{frame}

\begin{frame}{Leading Row Entry}
\begin{definition}
A \textbf{leading row entry} is the first left-most entry in a non-zero row that is non-zero.
\end{definition}
\begin{block}{Propositional Function}
\begin{itemize}
\item Given a m x n matrix A where m is the row size and n is the column size
\item Given some identifying row index i
\end{itemize}
\begin{equation}
\begin{split}
LRE(i, j) = \ &\left[\forall k \in \{ k | 0 <= k < j \}\left(A\left[i, k\right] = 0\right)\right] \\
            \ &\land A\left[i, j\right] \neq 0
\end{split}
\end{equation}
\end{block}
\end{frame}

\begin{frame}{Row Echelon Form}
\begin{definition}
\textbf{row echelon form} has the following three properties:
\begin{enumerate}
\item All rows containing all zeroes are at the bottom of the matrix
\item Each leading row entry is at least one column to the right of the leading entry above it
\item All entries in the column below the leading entry for a given row are zeroes
\end{enumerate}
\end{definition}
\end{frame}


\begin{frame}{Row Echelon Form: The first property}
\begin{block}{First Property of Row Echelon Form}
\begin{itemize}
\item Given a m x n matrix A where m is the row size and n is the column size
\end{itemize}
\begin{equation}
\begin{split}
&\forall i \in \{i | 0 <= i < m\}\Bigl[\Bigl(\forall j \in \{j | 0 <= j < n \}\left( A[i, j] = 0\right)\Bigr) \implies \Bigr. \\ 
&\left(\forall \Bigl. \in \{i^{\prime} | i < i^{\prime} < m\} \forall j^{\prime} \in \{ j^{\prime} | 0 < j^{\prime} < n \} \left(A[i^{\prime}, j^{\prime}] = 0\right) \right) \Bigr]
\end{split}
\end{equation}
\end{block}
\end{frame}

\end{document}