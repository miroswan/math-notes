\documentclass{beamer}
\usepackage{amsmath}

\title{Study Notes: Linear Algebra and its Applications}
\subtitle[short subtitle]{Systems of Linear Equations}
\author{Demitri Swan}
\beamertemplatenavigationsymbolsempty
\date{}

\begin{document}

\begin{frame}
  \titlepage % Generate the title slide
\end{frame}

\begin{frame}{Linear Equation}
\begin{definition}
Given variables $\left(x_{1}, ... , x_{n}\right)$\\
Given b and coefficients $\left(a_{1}, ... ,a_{n}\right)$ which are real or complex numbers\\
A \textbf{linear equation} can be written as:\\
\[ \left[ \sum_{i=1}^{n} a_{i}x_{i} \right] = b \]
\end{definition}
\end{frame}

\begin{frame}{System Of Linear Equations}
\begin{definition}
A \textbf{linear system} is a collection of linear equations in the same variables $\left(x_{1}, ... ,x_{n}\right)$ 
but including perhaps different cofficients. A linear equation can be written as:\\
\begin{align*}
\left[ \sum_{i=1}^{n} a_{i}x_{i} \right] &= b \\
\left[ \sum_{i=1}^{n} a^{\prime}_{i}x_{i} \right] &= b^{\prime}
\end{align*}
\end{definition}
\end{frame}

\begin{frame}{System Of Linear Equations}
\begin{example}
\begin{align*}
x_{1} - 2x_{2} &= 1\\
-x_{1} + 3x_{2} &= 3
\end{align*}
\end{example}
\end{frame}

\begin{frame}{Solution Of Linear Equations}
\begin{definition}
The \textbf{solution} of a system is a tuple of values $\left(s_{1}, ... , s_{n}\right)$ in $\left(x_{1}, ... ,x_{n}\right)$ 
that make each equation in the system true. A \textbf{solution set} is the set of all possible solutions. 
Linear systems are \textbf{equivalent} if they share the same solution set.
\end{definition}
\end{frame}

\begin{frame}{}
\end{frame}

\end{document}