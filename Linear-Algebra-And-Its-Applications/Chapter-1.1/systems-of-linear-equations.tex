\documentclass{beamer}
\usepackage{mathtools}
\usepackage{amsmath}

\title{Math Notes: Linear Algebra and its Applications}
\subtitle[short subtitle]{Systems of Linear Equations}
\author{Demitri Swan}
\beamertemplatenavigationsymbolsempty
\date{}

\begin{document}

\begin{frame}
  \titlepage % Generate the title slide
\end{frame}

\begin{frame}{Linear Equation}
\begin{definition}
Given variables $\left(x_{1}, ... , x_{n}\right)$\\
Given b and coefficients $\left(a_{1}, ... ,a_{n}\right)$ which are real or complex numbers\\
A \textbf{linear equation} can be written as:\\
\[ \sum_{i=1}^{n} a_{i}x_{i} = b_{i} \]
\end{definition}
\end{frame}

\begin{frame}{System Of Linear Equations}
\begin{definition}
A \textbf{linear system} is a collection of linear equations in the same variables $\left(x_{1}, ... ,x_{n}\right)$ 
but perhaps including different coefficients. A linear equation can be written as:\\
\begin{align*}
\sum_{i=1}^{n} a_{i}x_{i} &= b_{i} \\
\sum_{i=1}^{n} a^{\prime}_{i}x_{i} &= b^{\prime}_{i}
\end{align*}
\end{definition}
\end{frame}

\begin{frame}{System Of Linear Equations}
\begin{example}
\begin{align*}
x_{1} - 2x_{2} &= 1\\
-x_{1} + 3x_{2} &= 3
\end{align*}
\end{example}
\end{frame}

\begin{frame}{Solution Of Linear Equations}
\begin{definition}
The \textbf{solution} of a system is a tuple of values $\left(s_{1}, ... , s_{n}\right)$ in $\left(x_{1}, ... ,x_{n}\right)$ 
that make each equation in the system true. A \textbf{solution set} is the set of all possible solutions. 
Linear systems are \textbf{equivalent} if they share the same solution set.
\end{definition}
\end{frame}

\begin{frame}{Three Scenarios Regarding Solutions}
A solution to a linear system has:
\begin{itemize}
\item No solution (lines do not intersect)
\item One solution (lines intersect)
\item Infinite solutions (lines are coincident)
\end{itemize}
\end{frame}

\begin{frame}{Coefficient And Augmented Matrices}
\begin{definition}
  A \textbf{coefficient matrix} as a matrix containing only the coefficients of each equation in a linear system
  in the same order and position as the system itself. An \textbf{augmented matrix} has an additional column to 
  accommodate the numeric values on the right hand side of each equation in the system.
\end{definition}
\end{frame}

\begin{frame}{Coefficient And Augmented Matrices}
\begin{example}
\begin{equation}
\begin{split}
x_{1} - 2x_{2} &= 1 \\
-x_{1} + 3x_{2} &= 3
\end{split}
\end{equation}
\begin{equation}
\begin{bmatrix*}[r]
1 & -2 & 1 \\
-1 & 3 & 3 
\end{bmatrix*}
\end{equation}
\end{example}
\end{frame}

\begin{frame}{Elementary Row Operations}
The elementary row operations for solving a linear system:
\begin{itemize}
\item Replacement: Replace one row by the sum of itself and a multiple of another row
\item Interchange: Exchange two rows
\item Scaling: Multiply each element in a row by a non-zero constant
\end{itemize}
\textit{The goal is to get the matrix into reduced echelon form which is discussed in the next chapter}
\end{frame}


\begin{frame}{Row Equivalence}
\begin{definition}
Two matrices are row equivalent if there exists a sequence of elementary row
operations that transforms one matrix into the other.
\end{definition}
\end{frame}

\end{document}